%
% Проверьте, что вы сохранили файл в кодировке utf8
% encoding: utf-8
%

\documentclass[12pt]{report} % 12 -- размер шрифта

\usepackage{cmap} % Чтобы можно было копировать русский текст из pdf
\usepackage[T2A]{fontenc}
\usepackage[russian]{babel} % В частности эта строка отвечает за правильные переносы слов в конце строки
\usepackage[utf8]{inputenc} % Проверьте, что кодировка файла -- тоже utf8
\usepackage{amsmath, amssymb} % Чтобы юзать математические символы

\usepackage{hologo} % Логотип LaTex
\usepackage[russian]{hyperref} % http ссылки на внешние источники

\usepackage{longtable}
\usepackage{color}
\usepackage{setspace}

% Меняем размер листа, можно не менять
\textwidth=160mm
\hoffset=-15mm
\textheight=240mm
\voffset=-20mm

\newcommand{\Section}[1]{\section{#1}\vspace{-1.5em}\hspace*{\parindent}\unskip} % Фиксим багу с отступом в начале section, заодно пример собственных функций


\author{Захаренко Артем}
\title{Конспект конспекта по матану}
\date{\today}

\begin{document}
\maketitle
\tableofcontents
\newpage

\chapter*{Определенные интегралы}
\addcontentsline{toc}{chapter}{Определенные интегралы}

\section*{Обычные интегралы}
\addcontentsline{toc}{section}{Обычные интегралы}
$\displaystyle\int\limits_{a}^{b}{f(x)dx} = \sigma(f_{+}(x)) - \sigma(f_{-}(x))$

\noindent \textbf{Свойства}
\begin{itemize}
\item Аддитивность: $\displaystyle\int\limits_{a}^{b}{f(x)dx} = \displaystyle\int\limits_{a}^{c}{f(x)dx} + \displaystyle\int\limits_{c}^{b}{f(x)dx}$
\end{itemize}

\subsection*{\textbf{Теорема: монотонность интеграла}}
$f, g \in C[a, b], \ f(x) \leqslant g(x) \forall x \in [a, b] \Rightarrow \displaystyle\int\limits_{a}^{b}{f(x)dx}
\leqslant \displaystyle\int\limits_{a}^{b}{f(x)dx}$
\subsubsection{\textbf{Следствие:}}
$f \in C[a, b], (b - a)\min_{x \in [a, b]}(f(x)) \leqslant \displaystyle\int\limits_{a}^{b}{f(x)dx} \leqslant (b - a)\max_{x \in [a, b]}(f(x))$

\subsection*{\textbf{Теорема о среднем}}
$f \in C[a, b] \Rightarrow \exists c \in [a, b]: \displaystyle\int\limits_{a}^{b}{f(x)dx} = (b - a)f(c)$

\subsection*{\textbf{Определение:}}
Интеграл с переменным верхним пределом: $\Phi: [a, b] \rightarrow \mathbb{R}, \Phi(x) = \displaystyle\int\limits_{a}^{x}{f(t)dt}$

\subsection*{\textbf{Теорема Барроу: любая непрерывная на отрезке функция имеет первообразную}}
Основная идея: $R(y) = \frac{\Phi(y) - \Phi(x)}{y - x} = \frac{1}{y - x}\displaystyle\int\limits_{x}^{y}{f(t)dt} = f(c)$

\subsubsection*{\textbf{Следствие: на любом промежутке непрерывная функция имеет первообразную}}

\subsection*{\textbf{Теорема: формула Ньютона-Лейбница}}
$f \in C[a, b] \Rightarrow \displaystyle\int\limits_{a}^{b}{f(x)dx} = F(a) - F(b)$

\subsection*{\textbf{Теорема: линейность интеграла}}
$f, g \in C[a, b], \ \alpha, \beta \in \mathbb{R} \Rightarrow \displaystyle\int\limits_{a}^{b}{(\alpha f(x) + \beta g(x))dx} =  \alpha\displaystyle\int\limits_{a}^{b}{f(x)} + \beta\displaystyle\int\limits_{a}^{b}{g(x)}$

\subsection*{\textbf{Формулы интегрирования по частям и замена переменной}}
\begin{spacing}{1.5}
Приложение к формуле Валлиса:\\
\noindent $W_{n} = \displaystyle\int\limits_{0}^{\frac{\pi}{2}}{\sin^n(x)dx}$,\\
$W_{2k} = \frac{(2k - 1)!!}{(2k)!!}\frac{\pi}{2}$\\
$W_{2k + 1} = \frac{(2k)!!}{(2k + 1)!!}$\\
$W_{2k} \leqslant W_{2k + 1} \leqslant W_{2k + 2}$\\
$\Rightarrow \lim\left(\frac{(2n)!!}{(2n - 1)!! \cdot \sqrt{2n + 1}}\right) = \sqrt{\frac{\pi}{2}}$
\end{spacing}

\subsection*{\textbf{Теорема: Формула Тейлора с остатком в интегральной форме}}
\begin{spacing}{1.3}
$f \in C^{n + 1}\langle a, b  \rangle \Rightarrow \ \forall x \in \langle a, b \rangle \ 
f(x) = T_{n, x_{0}}{f(x)} + \frac{1}{n!}\displaystyle\int\limits_{x_0}^{x}(x - t)^nf^{(n + 1)}(t)dt$\\
Основная идея: индукция + интегрирование по частям
\end{spacing}

\subsection*{\textbf{Теорема Ламберта}}
\begin{spacing}{1.3}
$\pi$ и $\pi^2$ иррациональны\\
План доказательства:\\
$H_{j} = \frac{1}{j!}\displaystyle\int\limits_{0}^{\frac{\pi}{2}}{\left(\left(\frac{\pi}{2}\right)^2 - x^2\right)^j\cos x dx}$\\
\textbf{Свойства:}
\begin{itemize}
  \item $0 < H_j < \frac{1}{j!}\left( {\frac{\pi}{2}}\right)^{2j}$
  \item $\forall c > 0, \ H_jc^j \rightarrow 0$
  \item $H_{0} = 1, H_{1} = 2$
  \item $H_{j} = (4j - 2)H_{j - 1} - \pi^2H_{j - 2}$ (Интегрирование по частям)
  \item $\exists P_j, \ \deg(P_j) \leqslant j, \ H_j = P_j\left(\pi^2  \right)$
\end{itemize}
\end{spacing}
Дальше в 2 строки следует иррациональность $\pi^2$

\section*{Интегральные суммы}
\addcontentsline{toc}{section}{Интегральные суммы}

\subsection*{\textbf{Определение: равномерная непрерывность}}
$f : E \rightarrow \mathbb{R}$ равномерно непрерывна $\Rightarrow$ $\forall \varepsilon > 0 \ \exists \delta > 0: 
\forall x, y: |x - y| < \delta \Rightarrow |f(x) - f(y)| < \varepsilon$
\subsection*{\textbf{Теорема Кантора}}
\begin{spacing}{1.3}
$f: [a, b] \rightarrow \mathbb{R}$ непрерывна $\Rightarrow$ $f$ равномерно непрерывна.\\
Основная идея: от противного + строим последовательность и сходящуюся подпоследовательность\\
\end{spacing}

\newpage
\subsection*{\textbf{Определение: модуль непрерывности}}
\begin{spacing}{1.3}
$\omega_{f}(\delta) = \sup\{|f(x) - f(y)| : |x - y| < \delta \}$
\subsubsection*{\textbf{Свойства:}}
\begin{itemize}
  \item $\omega_{f}{(\delta)} \geqslant 0$
  \item $\omega_{f}{(0)} = 0$
  \item $\omega_{f}$ нестрого возрастает
  \item $ |f(x) - f(y)| \leqslant \omega_{f}{(|x - y|)}$
  \item $f$ --- липшецева с константой $M$ $\Rightarrow$ $\omega_{f}(\delta) \leqslant M\delta$
  \item $f$ --- равномерно непрерывна на $E$ $\Leftrightarrow$ $\omega_{f}$ непрерывна в 0
\end{itemize}
\end{spacing}

\chapter*{Анализ в метрических пространствах}
\addcontentsline{toc}{chapter}{Анализ в метрических пространствах}

\section*{Метрические и нормированные пространства}
\addcontentsline{toc}{section}{Метрические и нормированные пространства}
\subsection*{Определение:}

Метрика $\rho : X \times X \rightarrow [0, +\inf)$
\begin{enumerate}
  \setlength{\parskip}{0pt} % Отступ перед списком
  \setlength{\itemsep}{0pt} % Отступ между строками списка
  \item $\rho(x, y) = 0 \Leftrightarrow x = y$
  \item $\rho(x, y) = \rho(y, x)$
  \item Неравентсво треугольника $\rho(x, z) \leqslant \rho(x, y) + \rho(y, z)$
\end{enumerate}
Примеры:
\begin{enumerate}
  \setlength{\parskip}{0pt} % Отступ перед списком
  \setlength{\itemsep}{0pt} % Отступ между строками списка
  \item Дискретная метрика  
  
  \begin{equation*}
 \rho(x, y) =  
 \begin{cases}
   0, & x = y\\
   1, & x \neq y\\
 \end{cases}
\end{equation*}\\
 \item Манхетенская $\mathbb{R}^2$, $\rho((x_1, y_1), (x_2, y_2)) = |x_1 - x_2| + |y_1 - y_2|$
 \item $C[a, b]$ $\rho(f, g) = \int\limits_{a}^{b}{|f(x) - g(x)|dx}$\\
 		$\rho(f, h) = \int\limits_{a}^{b}(|f - h|) \leqslant \int\limits_{a}^{b}(|f - g| + |g - h|) =
 		\int\limits_{a}^{b}(|f - g|)  + \int\limits_{a}^{b}(|g - h|) = \rho(f, g) + \rho(g, h) $
 		
 	\item Расстояние в произвольном $n$-мерном пространстве.
\end{enumerate} 

\subsection*{Определение:}
$(X, \rho)$ --- метрическое пространство. \\ \\
$B_r(x) = \{y \in X : \rho(x, y) < r\}$ \\ \\
$\overline{B}_r(x) = \{y \in X : \rho(x, y) \leqslant r\}$
\subsubsection*{Свойства:}

\begin{enumerate}
\item $B_{r_1}(x) \cap B_{r_2}(x) = B_{\min(r_1, r_2)}(x)$
\end{enumerate}

\section*{Открытые и замкнутые множества}
\addcontentsline{toc}{section}{Открытые и замкнутые множества}


\subsection*{Определение: открытое множество}
$A \subset X$ --- открытое, если $\forall a \in A \ \ \exists r > 0 \ : \ B_r(a) \subset A$
\subsection*{Теорема об открытых множествах}
\begin{enumerate}
\item $X, \varnothing$ --- открытые
\item Объединение конечного числа открытых множеств --- открытое множество
\item Пересечение конечного числа открытых множеств --- открытое множество
\item $B_r(x)$ - открытое множество
\end{enumerate}

Доказательство:\\
\begin{enumerate}
\item Очев
\item Очев
\item Очев
\item $x \in B_R(a)$, берем $r = R - \rho(x, a)$. Докажем, что $B_r(x) \subset B_R(a)$. $y \in B_r(x) \Rightarrow \rho(x, y) < r = R - \rho(x, y) \Rightarrow R > \rho(x, y) + \rho(a, x) \geqslant \rho(a, y) \Rightarrow y \in B_R(a)$ 
\end{enumerate}

\subsection*{Определение: внутренняя точка}
Пусь $(X, \rho)$ - метрическое пространство, $A \subset X$, $a \in A$. Тогда $a$ - внутренняя $\Leftrightarrow$ $\exists \ r > 0 \ : \ B_r(a) \subset A$

\subsection*{Определение: внутренность множества}
Внутренность множества $Int A = \{a \in A : a$ ---	 внутренняя $\}$

\subsubsection*{Свойства:}
\begin{enumerate}
\item	$Int A \subset A$
\item $Int A = \cup_{C \subset A}C$, $C$ - открытые\\
Доказательство:

Пусть $B := \cup{A_i}$, $A_i$ --- все открытые подмножества $A$. Если $x \in B \Rightarrow \exists \ A_i : x \in A_i \Rightarrow \exists \ r > 0 : \  B_r(x) \subset A_i \subset A$. Если $x \in Int A \Rightarrow \exists r > 0 : B_r(x) \subset A$, $B_r(x)$ --- открытое множество.
\item $Int A	$ --- открыто
\item $Int A = A$ $\Leftrightarrow$ $A$ --- открыто
\item $B \subset A \Rightarrow Int B \subset A$
\item $Int (A \cap B) = Int A \cap Int B$ --- очев
\item $IntIntA = IntA$
\end{enumerate}

\subsection*{Определение: замкнутое множество}
$A$ --- замкнуто, если его дополнение (в $X$) открыто.

\subsection*{Теорема о свойствах замнкутых множеств}
\begin{enumerate}
\item $X, \varnothing$ --- замкнуты
\item Пересечение любого числа замкнутых множеств --- замкнутое (перетащим с открытых)
\item Объединене конечного числа замкнутых множеств --- замкнутое (перетащим с открытых)
\item $\overline{B}_r(a)$ --- замкнутое.
\end{enumerate}
Доказательство:
Тут все должно быть очевидно :)

\subsection*{Определение: замыкание множества}
Пересечение всех замкнутых множеств, содержащих данное --- его замыкание (обозначают за $CL(A)$)

\section*{Предельные точки}
\addcontentsline{toc}{section}{Предельные точки}

Точка $x$ предельная для множества $A$ $\Leftrightarrow$ если $\forall r > 0 B_r(x) \cap A \neq \varnothing$. $A^{\prime}$ --- множество предельных точек $A$.

\subsection*{Свойства предельных точек}

\begin{enumerate}
\item $CL(A) = A \cup A^{\prime}$
\item Если $A$ замкнуто, то $A^{\prime} \subset A$
\end{enumerate}

\subsection*{Теорема}
$x \in A^{\prime} \Leftrightarrow \forall r > 0 B_r(x)$ содержит бесконечно много точек из $A$. (ну это очев, из определения)
\section*{Подпространства}
\addcontentsline{toc}{section}{Подпространства}
$(X, \rho)$--- метрическое пространство\\
$Y \subset X$\\
$(Y, \rho|_{Y \times Y})$ --- подпространство
\subsection*{Теорема об открытых и замкнутых множествах в подпространстве}
$(X, \rho)$ --- метрическое пр-во $(Y, \rho)$ --- под-во, $A \subset Y$.
\begin{enumerate}
\item $A$ открыто в $Y$ $\Leftrightarrow$ $\exists G$ открытое в $X \ : A = G \cap Y$
\item $A$ замкнуто в $Y$ $\Leftrightarrow$ $\exists G$ замкнуто в $X \ : A = G \cap Y$
\end{enumerate}
\section*{Нормированные пространства}
\addcontentsline{toc}{section}{Нормированные пространства}
$X$ --- векторное пространство над $\mathbb{R}$.
$||x||$:$X \rightarrow \mathbb{R}$ --- 	норма, если:
\begin{enumerate}
\item $||x|| \geqslant 0 \forall x \in X$, $||x|| = 0 \Leftrightarrow x = 0_{X}$
\item $||\lambda x|| = \lambda||x|| \ \ \forall x \in X, \lambda \in \mathbb{R}$
\item $||x + y|| \leqslant ||x|| + ||y||$
\end{enumerate}
\subsection*{Скалярное произведение}
$\langle . \ . \rangle: X \times X \rightarrow \mathbb{R}$
--- скалярное произведение, если
\begin{enumerate}
\item $\langle x, x \rangle \geqslant 0, \ \langle x, x \rangle = 0 \Leftrightarrow x = 0$
\item $\langle x + y, z \rangle = \langle x, z \rangle + \langle y, z \rangle$
\item $\langle x, y \rangle = \langle y, x \rangle$
\item $\langle \lambda x, y \rangle = \lambda \langle x, y \rangle, \lambda \in \mathbb{R}$
\end{enumerate}

\subsection*{Свойства скалярного произведения и нормы}
\begin{enumerate}
\item Нер-во Коши-Буняковского $\langle x, y \rangle ^2 \leqslant \langle x, x \rangle \cdot \langle y, y \rangle$
Д-во:\\
$f(t) := \langle x + ty, x + ty \rangle \geqslant 0$\\
$\langle x, x \rangle + t\langle x, y \rangle + t\langle y, x \rangle + t^2 \langle y, y \rangle$ --- квдратный трехчлен больший 0, то есть $D \leqslant 0$. Дальше очев.
\item $||x|| = \sqrt{\langle x, x \rangle}$ --- норма. Все очев, нер-во треугольника по Коши-Буняковскому.
\item $\rho(x, y) = ||x - y||$ --- метрика. 
\item $||x - y|| \geqslant \mid ||x|| - ||y|| \mid$
\end{enumerate}

\section*{Пределы в метрических пространствах}
\addcontentsline{toc}{section}{Пределы в метрических пространствах}
$\{x_i\} \in X, \lim{x_i} = a$. 2 Варианта:
\begin{enumerate}
\item Вне любого $B_r(a), r > 0$ лежит конечное число точек из $\{x_i\}$
\item $\forall \epsilon > 0 \ \exists N: \ \forall n \geqslant N x_n \in B_{\epsilon}(a)$
\end{enumerate}
В общем-то, все как в $\mathbb{R}$. Стоит отметить, что и свойства такие же:
\begin{enumerate}
\item Если перемешать члены последовательности, то предел не изменится
\item У подпоследовательности такой же предел
\item Если каждую точку последовательности взять с конечной кратностью, то предел не изменится
\item Единственность предела\\

Доказательство:\\

Пусть у нас 2 предела $a$ и $b$, возьмем $r_a = r_b = \dfrac{\rho(a, b)}{3}$. Тогда вне $B_{r_a}(a)$ конечное число точек последовательности, вне $B_{r_b}(b)$ конечное число точек. Так как они не пересекаются, то во всей последовательности конечно число точек. Ну, так не бывает. 
\item \textbf{Новое свойство!} $a = \lim x_n \Leftrightarrow \lim \rho(x_n, a) = 0$
Стоит заметить, что оба выражения эквивалентны тому, что $\forall \epsilon > 0 \ \exists N : \ \forall n \geqslant N \rho(a, x_n) < \epsilon$
\end{enumerate}
\subsection*{Определение: ограниченное множество}
$A \subset X$ ограничено, если оно содержится в некотором шаре

\subsection*{Теорема: сходящаяся последовательность ограничена}
Очев же, как из первого семестра :)

\subsection*{Теорема: про предельные точки}
$a$ --- предельная точка $A$ $\leftrightarrow$ $\exists x_n \in A\setminus a, \lim(x_n) = a$. Писать много, но все тоже самое, что и в первом семе.

\subsection*{Теорема: про предельные точки}
$a$ --- предельная точка $A$ $\leftrightarrow$ $\exists x_n \in A\setminus a, \lim(x_n) = a$. Писать много, но все тоже самое, что и в первом семе.

\subsection*{Теорема: про арифметические действия}
$(X, \rho)$ --- нормированное пространство. $\lim(y_n) = y_0, \lim(x_n) = x_0$
\begin{enumerate}
\item $\lim(x_n + y_n) = x_0 + y_0$
\item $\lim(x_n - y_n) = x_0 - y_0$
\item $\lim(\lambda x_n) = \lambda\lim(x_n)$
\item $\lim(||x_n||) = ||x_0||$
\item Если в $X$ есть скалярное произведение, то $\lim \langle x_n, y_n \rangle = \langle x_0, y_0 \rangle$
\end{enumerate}

\subsection*{Частный случай: $R^d$}
Обычно норма это $||x|| = \sqrt{x_1^2 + ... + x_d^2}$. Для нее и рассматривают пределы. При этом, можно рассматривать сходимость по-координатно (то есть по каждой координает отдельно рассматривать последовательность из $\mathbb{R}$)\\

Не трудно видеть, что оба эти предела совпадают, так как пусть ${\rho(x_n, x_0) \rightarrow 0} \Rightarrow \left(x_n^{(1)} - x_0^{(1)}\right)^2 +  \left(x_n^{(2)} - x_0^{(2)}\right)^2 + ... + \left(x_n^{(d)} - x_0^{(d)}\right)^2 \rightarrow 0$. Сумма неотрицательных последовательностей стремится к 0, тогда каждая из них по отдельности стремится к 0. В другую сторону (от по-координатно к по норме) еще легче.
\subsection*{Определение: фундаментальная последовательность}
Определение почти как в первом семе)

\subsection*{Определение: полное пространство}
$(X, \rho)$ --- полное, если для любой фундаментальной последовательности она имеет предел
\subsection*{Теорема: $R^d$ --- полное}
Если $x_n$ --- фундаментальная, то $\forall \epsilon > 0 \ \exists N \forall m,n \geqslant N x_n - x_m< \rho(n, m) < \epsilon$
\end{document}
